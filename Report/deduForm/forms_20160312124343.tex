% Preamble
% ---
\documentclass{article}

% Packages
% ---
\usepackage{amsmath} % Advanced math typesetting
\usepackage{amsfonts} % Advanced math typesetting
\usepackage{amssymb} % More Math
\usepackage[utf8]{inputenc} % Unicode support (Umlauts etc.)
\usepackage[ngerman]{babel} % Change hyphenation rules
\usepackage{hyperref} % Add a link to your document
\usepackage{graphicx} % Add pictures to your document
\usepackage{listings} % Source code formatting and highlighting

\DeclareMathOperator*{\argmax}{arg\,max}


% Title
%---
\title{Question in Combining 3.1 with 3.2}
\author{Chang Li}
\date{\today}

% Document
%---
\begin{document}
	\maketitle
	\section{Fixed Space Formulation}
	$a_k=K(\theta_k-\theta_{k-1})$\\
	If $k =1$:
	$$
	b_1 = 0
	$$
	$$
	a_1 = K*b_1
	$$
	If $k \geq 2$
	\begin{align*}
	a_k&x_{k-1}+b_k= a_{k-1}x_{k-1}+b_{k-1}\\
	\frac{k-1}{K}&a_k+b_k=\frac{k-1}{K}a_{k-1}+b_{k-1}\\
	b_k &= -\frac{k-1}{K}a_k+\frac{k-1}{K}a_{k-1}+b_{k-1}\\
	&=-\frac{k-1}{K}a_k+\frac{k-1}{K}a_{k-1}-\frac{k-2}{K}a_{k-1}+\frac{k-2}{K}a_{k-2}+\dots-\frac{2-1}{K}a_2+\frac{2-1}{K}a_1+b_1\\
	&=-\frac{k-1}{K}a_k+\frac{1}{K}(a_{k-1}+\dots+a_1)+b_1
	\end{align*}
	Therefore, if $k \geq 2$:
	$$
	a_k*W(y)+b_k=a_k*W(y)+ (-\frac{k-1}{K}a_k)+\frac{1}{K}(a_{k-1}+\dots+a_1)+b_1
	$$
	\section{Random spaced samples}
	If we remove the fixed space constraint, sample $K$ breakpoints $\mathbf{h}=\{h_0,h_2,\dots,h_{K-1}\}$ with constraint $h_i>h_{i-1}$ from interval $[0,1]$ randomly. $h_0=0$ and $h_{K-1}=1$.\\
	$$a_k=(h_k-h_{k-1})(\theta_k-\theta_{k-1})$$
	If $k =1$:
	$$
	\theta_0=b_1 = 0
	$$
	$$
	a_1 =\frac{\theta_1-\theta_0}{h_1-h_0}= h_1*\theta_1
	$$
	If $k \geq 2$
	\begin{align*}
	a_k&h_{k-1}+b_k = a_{k-1}h_{k-1}+b_{k-1}\\
	b_k&= -h_{k-1}a_k+h_{k-1}a_{k-1}+b_{k-1}\\
	&=-h_{k-1}a_k+h_{k-1}a_{k-1}-h_{k-2}a_{k-1}+h_{k-2}a_{k-2}+\dots-h_{2-1}a_2+h_{2-1}a_1+b_1\\
	&=-h_{k-1}a_k+(h_{k-1}-h_{k-2})a_{k-1}+\dots+(h_2-h_1)a_2+(h_1-h_0)a_1+b_1
	\end{align*}
	Therefore, if $k \geq 2$:
	\begin{align*}
	a_k*W(y)+b_k=&a_k*W(y)+\\
	&(-h_{k-1})a_k+(h_{k-1}-h_{k-2})a_{k-1}+\dots\\
	&+(h_2-h_1)a_2+(h_1-h_0)a_1+b_1
	\end{align*}
	We can express the energy function with hidden variable $\mathbf{h}$ as $E(\mathbf{y},\mathbf{h};\mathbf{\theta})=\mathbb{\theta}^T\phi(\mathbf{y},\mathbf{h})$
	Where,
	\begin{equation*}
	\theta_k = \left\{
	\begin{aligned}
	& b_1	& \text{for} \ k=0\\
	& a_1 & \text{for}\ k=1\\
	& a_{k-1}-a_k  & \text{for} \ k=2,\dots,K\\
	\end{aligned}
	\right.
	\end{equation*}
	
	\begin{equation*}
	\phi_k = \left\{
	\begin{aligned}
	& 1	& \text{for} \ k=0\\
	& W(\mathbf{y}) & \text{for}\ k=1\\
	& \bigg(h_{k-1}-W(\mathbf{y}) \bigg)\bigg[\bigg[ W(\mathbf{y}) > h_{k-1}\bigg]\bigg]  & \text{for} \ k=2,\dots,K\\
	\end{aligned}
	\right.
	\end{equation*}
	
	\section{Combine 3.1 with 3.2}
	This is where I am confused. 
	Under the previous formulation, there is only pixel labels $\mathbf{y}$, namely $\mathbf{x}$ in \cite{yu2009learning} and ``latent variables'' $\mathbf{h}$ in linear function. My formulation is not a structural SVM. It 
	looks more like\cite{felzenszwalb2010object}. I couldn't find a way how to represent it with auxiliary variables $z$.
	
	
	\renewcommand\refname{Bibliography}
	\bibliographystyle{ieeetr}
	\bibliography{litera}
\end{document}