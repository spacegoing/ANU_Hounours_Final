% Preamble
% ---
\documentclass{article}

% Packages
% ---
\usepackage{amsmath} % Advanced math typesetting
\usepackage{amsfonts} % Advanced math typesetting
\usepackage{amssymb} % More Math
\usepackage[utf8]{inputenc} % Unicode support (Umlauts etc.)
\usepackage[ngerman]{babel} % Change hyphenation rules
\usepackage{hyperref} % Add a link to your document
\usepackage{graphicx} % Add pictures to your document
\usepackage{listings} % Source code formatting and highlighting

% Algorithm Packages
% ---
\usepackage{algorithm}
\usepackage[noend]{algpseudocode}
\makeatletter

% Defines
% ---
\def\BState{\State\hskip-\ALG@thistlm}
\makeatother
\DeclareMathOperator*{\argmax}{arg\,max}


% Title
%---
\title{Optimizing Weighted Lower Linear Envelope Potentials Within Latent-SVM Framework}
\author{Chang Li}
\date{\today}

% Document
%---
\begin{document}
	\maketitle
	\section{Algorithm}
	The higher-order energy function is:
	\begin{align*}
		E^c(y_c,z)&=a_1W_c(y_c)+b_1+\sum_{k=1}^{K-1}z_k((a_{k+1}-a_k)W_c(y_c)+b_{k+1}-b_k)\\
					&=a_1W_c(y_c)+\sum_{k=1}^{K-1}(a_{k+1}-a_k)z_kW_c(y_c)+\sum_{k=1}^{K-1}(b_{k+1}-b_k)z_k
	\end{align*}
	It can be written as $ E^c(y_c,z) = \theta^h\phi^h$ where
				\begin{equation}
				\label{param:hightheta}
					\theta^h_i = \left\{
					\begin{aligned}
						& a_1	& \text{for} \ i=1\\
						& a_i-a_{i-1} & \text{for}\ 1< i \leq K\\
						& b_{i+1-K}-b_{i-K} & \text{for} \ K<i\le2K-1\\
					\end{aligned}
					\right.
				\end{equation}
				
				\begin{equation*}
					\phi^h_i = \left\{
					\begin{aligned}
						& W(\mathbf{y}) 	& \text{for} \ i=1\\
						& W(\mathbf{y})\bigg[\bigg[i-1\le k^*\bigg]\bigg] & \text{for}\ 1<i\le K\\
						& \bigg[\bigg[ i-K\le k^*\bigg]\bigg]  & \text{for} \ K<i\le2K-1\\
					\end{aligned}
					\right.
				\end{equation*}
	Therefore, the energy function (higher order features together with unary and pairwise features) are:
	\begin{equation}
		E^{all}(y,z) = \begin{bmatrix}
		\theta^h\\
		\theta^{unary}\\
		\theta^{pairwise}
		\end{bmatrix}^T 
		\cdot \begin{bmatrix}
		\phi^h\\
		\phi^{unary}\\
		\phi^{pairwise}
		\end{bmatrix}=\theta^{all\text{T}}\cdot\phi^{all}
	\end{equation}
	where $\theta^{all}, \phi^{all} \in R^{2K+1}$.
	Therefore, we have the graph-cut method for inference latent variable:
	$$
	&\Delta((y_i,h_i^*(w)),(\hat{y}_i(w),\hat{h}_i(w)))\\
	\leq&\bigg(\max_{(\hat{y},\hat{h}) \in \mathcal{Y}\times\mathcal{H}}[w\cdot \phi(x_i,\hat{y},\hat{h})+\Delta(y_i,\hat{y},\hat{h})]\bigg)-\max_{h \in \mathcal{H}}{w\cdot\phi(x_i,y_i,h)}
	$$
	Let $\mathbf{z}^*_i(w) = \argmax_{\mathbf{z} \in \mathcal{Z}} w \cdot \Psi(\mathbf{y}_i,\mathbf{z})$
	$$
	(\mathbf{\hat{y}}_i(w),\mathbf{\hat{z}}_i(w))=\argmax_{(\mathbf{y} \times \mathbf{z}) \in \mathcal{Y} \times \mathcal{Z}} w\cdot\Psi(\mathbf{y},\mathbf{z})
	$$

					\begin{algorithm}
						\caption{MRF-LSSVM (CCCP-Cutting Plane)}\label{euclid}
						\begin{algorithmic}[1]
							\State \emph{Outer Loop}:
							\State $i \gets \textit{patlen}$
							\State \emph{top}:
							\If {$i > \textit{stringlen}$} \Return false
							\EndIf
							\State $j \gets \textit{patlen}$
							\BState \emph{loop}:
							\If {$\textit{string}(i) = \textit{path}(j)$}
							\State $j \gets j-1$.
							\State $i \gets i-1$.
							\State \textbf{goto} \emph{loop}.
							\State \textbf{close};
							\EndIf
							\State $i \gets i+\max(\textit{delta}_1(\textit{string}(i)),\textit{delta}_2(j))$.
							\State \textbf{goto} \emph{top}.
						\end{algorithmic}
					\end{algorithm}
\end{document}


